\documentclass[12pt]{article}

\usepackage{parskip}
\usepackage[margin=2.7cm,a4paper]{geometry} % Page formatting
\usepackage{multicol} % For multi-column sections
\usepackage{graphicx} % For including images
\usepackage{url}
\graphicspath{ {./images/} }
\usepackage{amsmath,amsthm,amsfonts,amssymb,mathtools} % For math symbols

% Math config
\DeclarePairedDelimiter\ceil{\lceil}{\rceil}
\DeclarePairedDelimiter\abs{\lvert}{\rvert}
\DeclarePairedDelimiter\set{\{}{\}}

% Header
\usepackage{fancyhdr}
\addtolength{\headheight}{2.5pt}
\pagestyle{fancy}
\fancyhead{} 
\fancyhead[L]{\sc INFO2222}
\fancyhead[C]{\sc bkni0201, kwal3204}
\fancyhead[R]{Project part 2: Usability}
\renewcommand{\headrulewidth}{0.75pt}

\usepackage[noend]{algpseudocode} % Pseudocode
\usepackage[tikz]{mdframed} % Outlined pseudocode

\newcommand{\bryce}{\hfill\normalsize\sc [bkni0201]}
\newcommand{\kai}{\hfill\normalsize\sc [kwal3204]}

\newcommand{\ekey}{\textsc{exchangeKey} }
\newcommand{\rkey}{\textsc{roomKey} }
\newcommand{\skey}{\textsc{storageKey} }

\newcommand{\func}[1]{\textsc{#1()}}

\begin{document}

\tableofcontents
\newpage

\section{Relation To Security Project}

Before beginning this stage of the project, there were some issues in the previous stage that we addressed at the beginning:

\textbf{The application breaks web crypto API specifications.} Upon login, we use \func{importKey} to import the user's password as a base cryptographic key, from which we derive \skey (described in the security project report). When importing this, the base key material was unnecessarily set \texttt{extractable=true}, which goes against the API specifications, but only caused an error in Google Chrome, rendering the application unusable. This was an easy fix, setting \texttt{extractable=false}.


\section{User Investigation}

\textbf{TODO in this section:}
\begin{itemize}
    \item Choose a user group, Students OR Alumni OR administrators
    \item Perform a PACT analysis on the chosen group
    \item Perform research on user group (surveys/interviews)
    \item Create a persona document for the user group
    \item Gather content of interest to target persona to be used for knowledgebase functionality examples (documents, articles)
\end{itemize}

\subsection{Best Practices - Accessibility \bryce}

Independently of any user investigations, there were a number of changes to make to the project for the sake of accessibility. This was done according to the Web Content Accessibility Guidelines (WCAG) published by the Web Accessibility Initiative (WAI) of the World Wide Web Consortium (W3C), making frequent referenct to the the WCAG2.2 quick reference. \cite{wcag-quickref}

In the context of this project, with the main userbase being university students, Accessibility is of vital importance, to both users who need alternative input and output devices such as Siip And Puff devices to emulate keyboard input, or screen readers. Since, at this time, we haven't performed any interactive userbase investigation, this was a good place to focus our efforts, as the requirements to make a website accessible are well-defined (and already known by myself from prior web experience).

We added appropriate \texttt{<label>} elements for forms, specifically on the login and signup pages. These function about the same as the paragraph items in the template, except they work properly with screen readers used by the visually impaired. A number of changes were made to the site for accessibility, including adding the \texttt{autofocus} property to the username input, and event listeners to focus on the next step upon presses of the enter key. This is in contrast to how users who cannot use a mouse (and developers quickly testing) navigate the page, using the tab and enter keys, which, on a site with a number of hyperlinks in the top navigation bar, quickly becomes tedious.

In the styling of the site, we kept accessibility in mind. We used  the html \texttt{rem} unit as the basis for the layout, so that the page would scale properly when zoomed in (as opposed to \texttt{vw} and \texttt{vh}). Pixels were used when setting the initial root font size, and \texttt{100dvh} was used as a minimum height for the main element. This use was justified as it does not impact the readability of the site, only potentially increasing the whitespace if the page content isn't large enough. We ensured that our choice of colors all contrasted well enough with each other in all their combinations.

Rem units are directly proportional to the height of one line of text. The concensus is that this is the best unit to use for elements' size, margins and padding, as the `size` of 1 pixel can vary between desktop and mobile devices, and not scale properly when zoomed in. Since rem units scale exactly with text, they will `look` the same no matter the individual client-side font settings, pixel scale or zoom level.

In choosing the colors fo rthe website, we used a tool called Realtime Colors \cite{rt-colors}. which is designed to randomly generate a coherent color scheme, and most importantly, one that complies with the WCAG recommended contrast ratios between colors for readability. The generated colors were used as a jumping-off point, and adjusted further beyond the initial number and types of colors, used, and re-checked to ensure we reached at least the minimum contrast level between colors. We set the font for the webpage to be Lexend, a sans-serif font designed specifically for maximum readability, defaulting to any sans-serif font if it cannot be found. Buttons were also added to the bottom of the page that switch between a defined light mode and dark mode color scheme, giving users the option to choose between the two.

\section{Design Plan} % Navigation Design in specs

\textbf{TODO in this section:}
\begin{itemize}
    \item Create "wishlist" of features
    \item conduct card-sorting with target users
    \item *maybe* since card sorting is for parts of the overall site, a section here for site-wide usability changes
    \item Outline results in report
    \item Using results, create a sitemap
\end{itemize}

\section[Prototype]{Prototype}

\textbf{TODO in this section:}
\begin{itemize}
    \item Using gathered information, brainstorm and create sketches of designs
    \item Create a prototype from the best design
    \item Perform a guerilla test, including group members and an outside person
    \item Outline guerilla test, including findings
    \item Make improvements to design plan from test findings
\end{itemize}

\section[Implementation]{Implementation Of Prototype}

\begin{itemize}
    \item Incremental Development Plan, each iteration 2 weeks
    \item Outline evaluations conducted
    \item Demonstrate functionality
    \item Self-evaluate, including team contribution
\end{itemize}

\subsection{META: Independent Features To Implement}

Due to the security project taking longer than expected, and the 1 week extension delaying its completion further, we simply didn't have enough time to wait for the design process to be completed before implementing new things. Following is a hastily made list of features and changes we were able to make to our code, independently of the results of the design process. These changes are either straight from the assignment specifications, or general web-development best practices.

\begin{itemize}
    \item New database table for articles
    \begin{itemize}
        \item Title, blurb, content attributes
        \item Comments go in new pooled database, with reference to article ID for easy modification
        \item Ensure proper content formatting, markdown, html, no XSS
        \item Database functions to edit, remove articles and comments
    \end{itemize}
    \item Implement architecture for multi-user chatrooms
    \begin{itemize}
        \item Rooms need to be decoupled from friendship id
        \item First 2 users perform exchange, agree on roomkey. each subsequent user generates an RSA pair and requests the room key.
        \item If we use RSA-OAEP from web crypto API, we get message integrity.
    \end{itemize}
    \item Implement staff ability to mute user from room or article (specs are unclear on exact meaning)
    \item New attributes for users, rank (student, staff, admin)
    \item List of online users (in python code, not database)
    \item HTML and CSS best practices
    \begin{itemize}
        \item Semantic html (main, nav, section, header instead of divs)
        \item Appropriate use of forms to enclose inputs
        \item label form inputs (helps people using screen readers)
        \item Ensure proper text formatting, size, tontrast (rems fopr sizing)
        \item Color code buttons for positive, negative, passive actions (e.g. green, gray, red)
        \item Add listeners for key presses, like enter on password entry
    \end{itemize}
\end{itemize}

%\newpage
%\section{DELETE THIS SECTION: Example code}
%
%\begin{figure}[h]
%    \caption{This is an example figure}
%    \centering
%    \includegraphics[width=0.75\textwidth]{oldAesFromPassword.png}
%\end{figure}
%
%This figure was generated using the codesnap \cite{CodeSnap} vscode extension, with the following settings in \texttt{settings.json}
%
%\begin{verbatim}
%"codesnap.showWindowControls": false,
%"codesnap.containerPadding": "0",
%"codesnap.roundedCorners": false,
%"codesnap.boxShadow": "none"
%\end{verbatim}

\newpage
\section{Bibliography}

In addition to the referenced sources, the following tools were used during this project:

\begin{itemize}
    \item Realtime Colors \cite{rt-colors}
\end{itemize}

\bibliographystyle{ieeetr}
\bibliography{refs}

\end{document}

